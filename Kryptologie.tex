% .:: Laden der LaTeX4EI Formelsammlungsvorlage
\documentclass[fs]{latex4ei}

% Dokumentbeschreibung
\title{Kryptologie EI SoSe 2011}
\author{Emanuel Regnath}
%\watermark{ \centering \put(50,-500){ \includegraphics {./img/watermarkDIN4.pdf} } }


%Kopf- und Fußzeile
\usepackage{fancyhdr}
\pagestyle{fancy}
\fancyhf{}

   \fancyfoot[C]{von Emanuel Regnath (Emanuel.Regnath@tum.de)}
   \renewcommand{\headrulewidth}{0.0pt} %obere Linie ausblenden
   \renewcommand{\footrulewidth}{0.1pt} %obere Linie ausblenden

   \fancyfoot[R]{Stand: \today \ um \thistime \ Uhr \qquad \thepage}
   \fancyfoot[L]{Homepage: emareg.de -  Fehler bitte sofort melden.}
	

\newcommand{\GF}{\ensuremath{\mathrm{GF}}}									%ggT
	
	
% Dokumentbeginn
% ======================================================================
\begin{document}


% Aufteilung in Spalten
\begin{multicols}{4}

% -------------------------------------------
% | 		KRYPTOLOGIE						|
% ~~~~~~~~~~~~~~~~~~~~~~~~~~~~~~~~~~~~~~~~~~~
%=======================================================================



% SECTION ====================================================================================
\section{Kryptologie - Allgemeines}
% ============================================================================================
	\begin{tabular}{@{}ll}
	Kryptographie & Entwicklung und Anwendung von Verschlüsselungen.\\
	Kryptoanalyse & Testen und knacken von Verschlüsselungen.\\
	Steganographie & Verbergen von geheimen Botschaften.\\ 
	%LSBs von digitalen Daten manipulieren.
	\end{tabular}

	\subsection{Entwicklung von Algorithmen}
	Bewährtes Vorgehen: Öffentliche Entwicklung von Algorithmen. Große Zahl von Schlüsseln. Nur Schlüssel geheim.\\
	\begin{tabular}{ll}
		symmetrisch & gleicher Schlüssel für Ver- und Entschlüsselung.\\
		asymmetrisch & öffentlicher und privater Schlüssel.
	\end{tabular}
	
	\subsection{Kryptoanalyse}
	\begin{tabular}{ll}
		Brute-Force & Vollständiges Durchsuchen des Schlüsselraums.\\
		Ciphertext-only & einer oder mehrere Chiffretexte sind bekannt.\\
		Known-plaintext & Einige Chiffre wie auch Klartext bekannt.\\
		Chosen-plaintext & wählbarer zu verschlüsselnder Text.\\
		Häufigkeitsanalyse & Buchstabenhäufigkeit einer Sprache.
	\end{tabular}


	\subsection{perfekte Sicherheit}
	ist gewährleistet, wenn $|M| = |C| = |K|$
	die Länge der Nachricht, des Chiffrats und des Schlüssels gleich lang ist und alle Chiffre gleich wahrscheinlich sind.


	\textbf{Kerckhoffs'sches Prinzip:} Die Sicherheit eines Kryptosystems darf nicht von der Geheimhaltung des Algorithmus abhängen, sondern nur von der Geheimhaltung des Schlüssels!
	\\
	Shannon:\\
	Große Konfusion: Jede Chiffrekombination soll gleich wahrscheinlich auftreten\\
	Große Diffusion: Jede kleine Änderung am Klartext oder Schlüssel soll eine große Änderung im Chiffre bewirken.
	Sicherheit gegen bekannte Angriffe.\\


% SECTION ====================================================================================
\section{Mathematische Grundlagen}
% ============================================================================================

Euklidscher Satz: $a = q \cdot n + r$ \qquad $r = a \mod n$\\
$a \mod n = a \mod (-n)$\\
$a + i\cdot n \mod n = a \mod n = r$\\
$(a_1 + a_2) \mod n = (a_1 \mod n + a_2 \mod n) \mod n$\\
$(a_1 \cdot a_2) \mod n = (a_1 \mod n \cdot a_2 \mod n) \mod n$\\
		
	\subsection{größter gemeinsamer Teiler $\ggT$}
	$\mathrm{ggT}(a,0) = |a|$ \quad	$\ggT(a+i\cdot n,n) = \ggT(a,n)$\\		
	Rekursive Definition: $\ggT(a,n) := \ggT(n,a \mod n)$\\

	
	\subsection{Algebraische Systeme}
	Eine Algebraische Struktur ist eine Menge $S$ mit einer oder mehreren binären Verknüpfungen $*:S\times S \rightarrow S$, die bestimmte Axiome erfüllen.\\
	\begin{tabular}{l|p{3.5cm}}
		Struktur & Definition \\ \hline
		Halbgruppe $(S,*)$ & $*$ ist assoziativ \\
		Monoid $(S,*)$ & Halbgruppe mit neutralem Element $e$\\
		Gruppe $(S,*)$ & Monoid mit Inversem $^{-1}$\\
		Abelsche Gruppe $(S,*)$ & Gruppe, so dass $*$ kommutativ ist.\\
		Ring $(S,+,\cdot)$ & $(S,+)$ ist abelsche Gruppe, $(S,\cdot)$ ist Halbgruppe 
			und es gilt das Distributivgesetz \\
		Körper(Field) $(S,+,\cdot)$ & Ring, so dass $(S\setminus\{ 0 \}, \cdot)$ abelsche Gruppe ist. (Multiplikativ Inverse)\\
	\end{tabular}\\
	Ordnung der Gruppe $\# \N_n = n$\\
	
	Kommutativer Ring: 
	abelsche Gruppe $\langle S,\oplus \rangle$
	und Monoid $\langle S,\odot \rangle$
	
	\subsection{Modulo-Arithmetik}
	Additiv Inverses $-a + a = 0 \mod n$\\
	Multiplikativ Inverse $a^{-1} \cdot a = 1 \mod n$\\ 
	$a^{-1}$ existiert nur, wenn $\ggT(a,n) = 1$\\

	$(a_1 + a_2) \mod n = ( a_1 \mod n + a_2 \mod n) \mod n$\\
	$(a_1 \cdot a_2) \mod n = ( a_1 \mod n \cdot a_2 \mod n) \mod n$\\
	??$(a_1 \cdot a_2) \mod n = (n-r) \cdot (n-r) \mod n$??

	\subsection{Polynome}
	Der Galois-Körper $GF(p) = \langle \mathbb Z_p, + \mod p, \cdot \mod p \rangle$ ist besonders wichtig.\\
	$GF(2) = \R_n[x]$ mit Koeffizienten $a_i = \eset{1,0}$\\
	Andere Darstellung: $x^3 + x - 1 = 1 x^3 + 0 x^2 + 1x^1 + 1x^0 = 1011$\\
	
	
		\subsubsection{Division $A(x) : M(x) = C(x) + R(x)$}
		Immer höchste 1 von $A(x)$ mit XOR $M(X)$ auslöschen, dann Positionen der $M(x)$ zählen.\\
		\begin{tabular}{lllllllll}
		& 10011 & : & 101 = 101 R 10\\
		$\oplus$ & 101 & & \\ \mrule
		 & 00111 & & \\
		$\oplus$ & \quad 101 & \\ \mrule
		 & \quad 010
		\end{tabular}
		
		\subsubsection{Modulo Polynom $A(x) \mod M(x) = R(x)$}
		Wie Polynomdivision, der Rest $R(x)$ ist gesucht.\\ 
		$M(x)$ solange mit XOR durchschieben bis Polynom kleiner $M(x)$\\
		$1101 \mod 11 = 1$, \quad $10011 \mod 101 = 10$\\
	
		
	Nullstellenfreiheit bedingt nicht irreduzibilität.\\

	Falls Polynom gerade $\Ra$ Wurzel ziehen\\
	\\
	\subsection{Quadratzahlen}
	Quadratzahlen: $y = a^x \mod p$ \\
	Zu jedem Quadratischen Rest gibt es zwei mögliche Wurzeln\\
	Jacobitest: $J(a,p) = a^{\frac{p-1}{2}} \mod p$\\
	$J(a,p) = 1 \mod p \Ra$ Quadratischer Rest \\
	$J(a,p) = -1 \mod p \Ra$ Quadratischer NichtRest (keine Wurzel)\\
	Trick: $a = \pm a^{\frac{p+1}{2}} = \pm (a^2)^{\frac{p+1}{4}} \mod p$ falls $p \mod 4 = 3$\\
	Falls $p$ nicht Prim, dann Durchlauf für jede Teilprimzahl. Ergebnis miteinander multiplizieren. 
	2 Teilprimzahlen $\Ra$ 4 Ergebnisse


	\subsection{Euklidischer Algorithmus}
	Euklidischer Algorithmus: $\ggT(544,323) = 17$ bestimmen.\\
	\begin{tabular}{ll}
		$544 =$ & $1 \cdot 323 + 221$\\
		$323 =$ & $1 \cdot 221 + 102$\\
		$221 =$ & $2 \cdot 102 + 17$\\
		$102 =$ & $6 \cdot \boldsymbol{17} + 0$\\
	\end{tabular}
	\subsection{Erweiterter Euklidischer Algorithmus}	
	wird benutzt um die multiplikative Inverse $a^{-1}$ zu bestimmen\\
	$\ggT(a,n) = v_{\ir end} a + u_{\ir end} n$\\	
	Bestimme $42^{-1} \mod 97$ \qquad Berechne $\ggT(42,97)$\\
	%$1_1 = \floor{\frac{92}{42}} = 2$\\
	\begin{tabular}{@{}rll@{}}
		$97 =$ & $1 \cdot 97$ & $+ 0 \cdot 42$\\
		$42 =$ & $0 \cdot 97$ & $+ 1 \cdot 42$\\ \mrule
		$97 - 2\cdot 42 =$ & $(1-2\cdot 0) \cdot 97$ & $+ (0 - 2 \cdot 1) 42$\\
		$13 =$ & $1 \cdot 97$ & $-2 \cdot 42$\\ \mrule
		$42 - 3 \cdot 13 =$ & $(0 - 3 \cdot 1) \cdot 97$ & $(1 + (-3)(-2)) \cdot 42$\\  
		$3 =$ & $-3 \cdot 97$ & $+ 7 \cdot 42$\\  \mrule
		$13 - 4 \cdot 3 =$ & $(1 - 4(-3)) \cdot 97$ & $+ (-2- 4 \cdot 7) \cdot 42$\\  
		$1 =$ & $\underset{=0}{13 \cdot 97}$ & $\boldsymbol{- 30} \cdot 42$\\  
	\end{tabular}
	
	 
	
	
% SECTION ====================================================================================
\section{Sicherheitsdienste}
% ============================================================================================
	\begin{itemize}\itemsep0pt
		\item Verbindlichkeit, Nachweisbarkeit (Nicht-Abstreitbarkeit)
		\item Authentifikation (Identitätsnachweis)
		\item Integrität (Unverändert?)\vspace{0.4em}
		\item Vertraulichkeit (Geheimhaltung)
		\item Anonymität\vspace{0.4em}
		\item Berechtigung (Zugangskontrolle)
	\end{itemize}
	
	
	

	\subsection{OSI Sicherheitsarchitektur}
	\begin{tabular}{l|c|l}
		Was kann passieren? & \framebox{Bedrohungen} & Abhören,Manipulation\\
		Sicherheitsgrad? & \framebox{Regeln} \\
		Gegenmaßnahmen? & \framebox{Sicherheitsdienste} & Vertraulichkeit\\
		Hilfmittel? & \framebox{Mechanismen} & Verschlüsselung\\
		Verfahren? & \framebox{Algorithmen} & AES\\
	\end{tabular}


	\subsection{MAC -- Message Authentification Code}
	64,128 oder 160 bit Wert um Integrität und Authentizität nachweisen. \\
	Initwert $I$, $MAC = g(m,k, g(m,k,...))$\\
	
	\subsection{Einwegfunktionen}
	Bsp: Multiplikation von zwei großen Primzahlen.

	\subsection{Krypto-Hash-Funktionen}
	Nicht bijektive Einwegfunktion um lange Nachrichten auf kurze Hashwerte abzubilden.
	 $h(m)$


% SECTION ====================================================================================
\section{Verschlüsselungsverfahren}
% ============================================================================================

Verschlüsselung ist eine bijektive Abbildung: $f:M \leftrightarrow C$\\
$m$: Klartext, $c$: Chiffre, $k$: Schlüssel\\
symmetrisch: $c=f(k,m)$ \qquad $m=f^{-1}(k,c)$\\
Anzahl der Schlüssel bei $n$ Teilnehmern: $|K| = \frac{n(n-1)}{2}$\\
asymmetrisch: $c=f_e(m)$ \qquad $m=f_d(c)$\\
Sender \textbf{ver}schlüsselt mit \textbf{fremden} öffentlichen Schlüssel. Empfänger \textbf{ent}schlüsselt mit \textbf{eigenem} privaten Schlüssel.\\




	\subsection{Permutationsalgorithmen}
	Vertauschen der Buchstabenpositionen in einem Text. 
		
		\subsubsection{Skytala}
		500 v. Chr. in Sparta: Aufwickeln eines Gürtels auf einen Stock.\\
	

	\subsection{Substitutions-Algorithmen}
	Jeweils ein Klartextbuchstabe wird durch einen Chiffrebuchstaben ersetzt.\\
	Kann oft durch Häufigkeitsanalyse (HA) gebrochen werden.

		\begin{tabular}{lll}
			Ceasarcode & 1 Fester Verschiebungsschlüssel & $26$\\
			Vigenère-Chiffre & $r$ periodische Schlüssel & $26^r$\\
			Verman-Chiffre & binär, einmalig, Strom-Chiffre & $\infty$\\
			Enigma & 3 aus 5 Rotationsscheiben & $~10^{23}$
		\end{tabular}
		
	
	Caesar: $C_k = in + k \mod 26$\\
	Caesar arbeitet auf einer Gruppe $\langle \Z_26, + \mod 26 \rangle$\\
	Vigenère: Polyalphabetische Substitution.\\
		Angriff: Wiederholung von Buchstabenfolgen, Primfaktoren ergeben Schlüssellänge, dann HA\\
		Schlüssellänge $h$ für Koinzidenzindex $k$: $h \approx (k_d - k_r) n / [(n-1) k - k_rn + k_d]$



	\subsection{Blockchiffre}
	Blockweise Verschlüsselung.
	Typische Blocklänge: 64Bit, typische Schlüssellänge 128Bit -- 256Bit.\\
	
	
		\subsection{DES}
		Digital Encryption Standard
		64-Schlüssel bestehend aus 56 Bit + 8 Parity Bits\\
		Permutation und Shifts nötig, damit jedes Schlüsselbit jedes Textbit beeinflussen kann.
		Insgesamt 28 Bit Schlüsselshifts
	
			F-Funktion: Expansion von 32 Textbits auf 48 durch Bitverdopplung.
			Aufteilung in 8 mal 6 Bit: S-Funktion: 1. und 6. Bit bestimmen Zeile:
			\begin{tabular}{l|lll}
				& 0000 & 0001 & ...\\ \hline
				00 & \\
				01 & \\
				10 & \\
				11 & \\
			\end{tabular}
			S-Bos ist einzige nichtlinearität\\
			S-Box hat 6 inputs und 4 outputs
	
		Verschlüsselung: $R_{16} = L_{15} \otimes F(R_{15}, k_{16})$
	
	
		\subsubsection{IDEA}
	
	
	
		\subsubsection{AES}
		Blocklänge: 128Bit; Schlüssellänge: 128Bit - 256Bit.\\
		Modulpolynom $M(x) = x^8 + x^4 + x^3 + x + 1 = \eset{100011011}$
		Substitution ist die einzige nichtlineare Operation.\\
	
	
	
		\subsubsection{Block-Operations-Modi}
		\begin{itemize}
			\item ECB (electronic Codeblock): Jeder Block wird unabhänig verschlüsselt. Sollte nur für Nachrichten $< 1$ Block verwendet werden.
			\item CBC (Cipher Block Chaining): Cipherblock wird mit nächsten Klartextblock verkettet. (XOR)
			\item CFB (Cipher Feedback): Selbstsynchronisierende Stromchiffre.\\
			\item OFB (Output Feedback): keine Selbssync.
			\item CTR (Counter):
		\end{itemize}
		nonce: number used only once
	
	
	\subsection{Stromchiffre}
	Zeichenweise Verschlüsselung. z.B. XOR: $c_i = m_i \otimes k$ \quad $m_i = c_i \otimes k$\\
	Kryptoanalyse: Chiffre mit 0-Folge.


% SECTION ====================================================================================
\section{Asymmetrische Verfahren}
% ============================================================================================
	\subsection{Potenzen in Arithmetik modulo n}
	$y = a^x \mod p$\\
	\begin{tabular}{c|lllll}
	$x\backslash q$ & 0 & 1 & 2 & 3 & 4 \\ \mrule
	1 & 0 & 1 & 2 & 3 & 4 \\
	2 & 0 & 1 & 4 & 4 & 1 \\
	3 & 0 & 1 & 3 & 2 & 4 \\
	4 & 0 & 1 & 1 & 1 & 1 \\
	\end{tabular}\\
	\\
	Kleiner Satz von Fermat: \boxed{ a^{p-1} \mod p = 1 }, falls $\ggT(a,p)=1$\\
	$a = 2 \lor a = 3$ sind Generatorelemente.\\
	\\
	Eulersche $\Phi$-Funktion: $\Phi(n) = | \iset{ z \in [1,n-1]}{\ggT(n,z) = 1} |$\\
	Es gilt $\Phi(p) = p-1$ und $\Phi(p \cdot q) = (p-1)(q-1)$ für $p \ne q$\\
	Euler's Satz: $a^{\Phi(n)} \equiv 1\ (\!\!\!\!\mod n)$ \qquad $a^{k \Phi(n) + 1} \equiv a\ (\!\!\!\!\mod n)$\\ 
	$a^j \mod n = a^{j \mod \Phi(n)} \mod n$\\


	\subsection{Fermat-Test für $n$}
	Falls für jedes $a < n$ mit $\ggT(a,n) = 1$ gilt $a^{n-1} \ne 1 \mod n$, dann
	ist $n$ vermutlich prim 

	
	\subsection{Miller-Rabin-Test}
	$(n-1) / a^s = d \in \N$ \quad größtes $s$, so dass $d \in \N$\\
	Falls $n$ prim, dann entweder $a^d = l \mod n$ \\
	oder $\exists r \le s-1: a^{2^r d} = -1$\\
	
	\subsection{Chinesischer Restsatz}
	Sind zwei Reste $a = x \mod p$ und $b = x \mod q$ mit $n = pq$ bekannt, dann kann
	$x \mod n$ berechnet werden:
	
	
	
	\subsection{RSA}
	Erzeugung eines Schlüsselpaares: 
	\begin{enumerate}
		\item $n = p \cdot q$ mit $p \ne q$
		\item Wähle ersten Schlüssel $e$ zufällig mit $1 < e < \Phi(n)$ und $\ggT(e,\Phi(n)) = 1$
		\item Berechne privaten Schlüssel $d$ durch $e \cdot d \equiv 1 \mod \Phi(n)$
	\end{enumerate}	
	
	\begin{tabular}{ll}
	Verschlüsselung & $c = (m^e) \mod n$\\
	Entschlüsselung & $(c^d) \mod n = (m^e)^d \mod n = m$\\
	Signatur & $s = (m^d) \mod n$\\
	Verifikation & $(s^e) \mod n = (m^d)^e \mod n = m$\\
	\end{tabular}
		
		\subsection{Diffie-Hellman-Schlüsselvereinbarung (1976)}
		Protokoll um öffentlich einen geheimen Schlüssel $k$ zu vereinbaren. Bietet jedoch keine Authentifizierung.
		Vereinbarung kann öffentlich abgehört werden ohne Sicherheit zu gefährden.\\
		\\
		Beide Partner Alice und Bob vereinbaren eine 1024Bit Primzahl $p$ und eine Basis $g \in \GF(p)$ daraus wird der gemeinsame Schlüssel $k$ abgeleitet.\\
		\begin{tabular}{l|l}
			Alice & Bob\\
			wählt geheime Zufallszahl $a$ & wählt geheime Zufallszahl $b$\\
			berechnet $\alpha = g^a \mod p$ & berechnet $\beta = g^b \mod p$\\
			schickt $\alpha$ an Bob & schickt $\beta$ an Alice\\
			berechnet $k = \beta^a \mod p$ & berechnet $k = \alpha^b \mod p$\\
		\end{tabular}
		Es gilt $\beta^a = \alpha^b = g^{ba}$	


		\subsection{ElGamal-Verfahren (1984)}
		Öffentlich bekannt: Primzahl $p$ und Basis $g \in \GF(p)$\\
		Jeder Teilnehmer wählt einen privaten Schlüssel $d$ und berechnet daraus den öffentlichen Schlüssel $e = g^d \mod p$


	\subsection{Elliptische Kurven, ECC Kryptographie}
	Befinden sich auf einem 2D Galios Field (Torus)\\
	Ist unendlich lang\\
	Sicherheitsvergleich: 160Bit ECC entspricht 1024Bit RSA.\\
	Elliptische Kurve: \boxed{y^2 = x^3 +ax + b}
	Diskrete Punkte ergeben sich aus Schnittpunkten zwischen der el. Kurve und einer Gerade. Die Koordinaten
	der Schnittpunkte müssen ganzzahlig sein.



% SECTION ====================================================================================
\section{Hashfunktionen}
% ============================================================================================
Nicht Kryptographisch: Datenbanken\\
Kryptographisch: Passwörter, Fingerabdruck, Zertifikat:\\
	keyed (MAC) and unkeyed (MDC)\\
\\
Anforderungen (Resistance):
\begin{description}
	\item[Preimage] Einweg: schwierig zu $h$ ein $m$ mit $h = h(m)$ zu finden
	\item[2nd Preimage] schwierig $m_2$ mit $h(m_2) = h(m_1)$ zu finden
	\item[Collision:] schwierig $m_1,m_2$ mit $h(m_2) = h(m_1)$ zu finden
\end{description}

\subsection{MD5}


\subsection{SHA}


\subsection{Keccak25}
Neue Hashfunktion mit anderem Ansatz als bestehende Hashfunktion(SHA,MD5)\\
3D: state, 2D: plane, slice, sheet, 1D: row, column, lane 0D: bit\\
$R = i \xi \pi \rho \theta$\\
$\theta:$ Ausummieren der Säulen zu einem Bit:\\
$\rho:$ Sheet-Permutation\\
$\pi:$ Spaltenpermutationsmatrix \\
$\xi:$ Lanepermutation: $L_1' = L_{1} \oplus (\neg L_2 \land L_3)$\\
\\
Bitrate $r$, Kapazität $c$, statesize $b = r+c$





\subsection{Protokolle}
Definition: Satz von Regeln für den Austausch von Daten zwischen Kommunikationspartnern.\\

Fiat-Shamir Authentifizierung:\\
Schlüsselbank: $n = p \cdot q$; $n$ öffentlich; $p,q$ geheim\\
Für jeden Teilnehmer Zufallszahl $z_i$ und Geheimnis $s_i$


Kerberos (sym.)
Trustet Third Party TTP:\\

Needham Schroeder Protokoll (asym.)



Square and Multiply (L to R) Algorithmus:\\



\section{IT-Sicherheit}
	- Safety: Sicher für den Menschen bei Fehlfunktion.\\
	- Secure: Sicher gegen Angriffe.\\
	
	\subsection{Datensicherheit}
	
	

	Quantencomputer: Faktorisierungsprobleme lösbar -> RSA unsicher
					AES 128 unsicher, 256Bit noch als sicher erachtet.\\





	\subsection{Symbole}
	\begin{tabular}{lll}
		Primzahl $p,q$ & Nat. Zahl $n,m$\\
		Message $m$ & Cyphertext $c$\\
		priv. key $d$ & pub. key $e$\\
	\end{tabular}




% Ende der Spalten
\end{multicols}



% Dokumentende
% ======================================================================
\end{document}









